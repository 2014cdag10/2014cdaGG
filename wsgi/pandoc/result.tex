\documentclass[]{article}
\usepackage[T1]{fontenc}
\usepackage{lmodern}
\usepackage{amssymb,amsmath}
\usepackage{ifxetex,ifluatex}
\usepackage{fixltx2e} % provides \textsubscript
% use upquote if available, for straight quotes in verbatim environments
\IfFileExists{upquote.sty}{\usepackage{upquote}}{}
\ifnum 0\ifxetex 1\fi\ifluatex 1\fi=0 % if pdftex
  \usepackage[utf8]{inputenc}
\else % if luatex or xelatex
  \ifxetex
    \usepackage{mathspec}
    \usepackage{xltxtra,xunicode}
  \else
    \usepackage{fontspec}
  \fi
  \defaultfontfeatures{Mapping=tex-text,Scale=MatchLowercase}
  \newcommand{\euro}{€}
\fi
% use microtype if available
\IfFileExists{microtype.sty}{\usepackage{microtype}}{}
\usepackage{color}
\usepackage{fancyvrb}
\newcommand{\VerbBar}{|}
\newcommand{\VERB}{\Verb[commandchars=\\\{\}]}
\DefineVerbatimEnvironment{Highlighting}{Verbatim}{commandchars=\\\{\}}
% Add ',fontsize=\small' for more characters per line
\newenvironment{Shaded}{}{}
\newcommand{\KeywordTok}[1]{\textcolor[rgb]{0.00,0.44,0.13}{\textbf{{#1}}}}
\newcommand{\DataTypeTok}[1]{\textcolor[rgb]{0.56,0.13,0.00}{{#1}}}
\newcommand{\DecValTok}[1]{\textcolor[rgb]{0.25,0.63,0.44}{{#1}}}
\newcommand{\BaseNTok}[1]{\textcolor[rgb]{0.25,0.63,0.44}{{#1}}}
\newcommand{\FloatTok}[1]{\textcolor[rgb]{0.25,0.63,0.44}{{#1}}}
\newcommand{\CharTok}[1]{\textcolor[rgb]{0.25,0.44,0.63}{{#1}}}
\newcommand{\StringTok}[1]{\textcolor[rgb]{0.25,0.44,0.63}{{#1}}}
\newcommand{\CommentTok}[1]{\textcolor[rgb]{0.38,0.63,0.69}{\textit{{#1}}}}
\newcommand{\OtherTok}[1]{\textcolor[rgb]{0.00,0.44,0.13}{{#1}}}
\newcommand{\AlertTok}[1]{\textcolor[rgb]{1.00,0.00,0.00}{\textbf{{#1}}}}
\newcommand{\FunctionTok}[1]{\textcolor[rgb]{0.02,0.16,0.49}{{#1}}}
\newcommand{\RegionMarkerTok}[1]{{#1}}
\newcommand{\ErrorTok}[1]{\textcolor[rgb]{1.00,0.00,0.00}{\textbf{{#1}}}}
\newcommand{\NormalTok}[1]{{#1}}
\ifxetex
  \usepackage[setpagesize=false, % page size defined by xetex
              unicode=false, % unicode breaks when used with xetex
              xetex]{hyperref}
\else
  \usepackage[unicode=true]{hyperref}
\fi
\hypersetup{breaklinks=true,
            bookmarks=true,
            pdfauthor={機械設計工程系二甲},
            pdftitle={2014 協同產品設計實習報告},
            colorlinks=true,
            citecolor=blue,
            urlcolor=blue,
            linkcolor=magenta,
            pdfborder={0 0 0}}
\urlstyle{same}  % don't use monospace font for urls
\setlength{\parindent}{0pt}
\setlength{\parskip}{6pt plus 2pt minus 1pt}
\setlength{\emergencystretch}{3em}  % prevent overfull lines
\setcounter{secnumdepth}{0}

\title{2014 協同產品設計實習報告}
\author{機械設計工程系二甲}
\date{April 23, 2014}

 
\usepackage{xeCJK}    % 中英文字行分開設置 
\usepackage[T1]{fontspec}    %設定字體用 
\usepackage{graphicx} 
\usepackage{fancyvrb} % for frame on Verbatim 
\setCJKmainfont{新細明體}
\begin{document}
\maketitle

{
\hypersetup{linkcolor=black}
\setcounter{tocdepth}{3}
\tableofcontents
}
\section{前言}\label{ux524dux8a00}

協同產品設計實習課程目標\footnote{這是註解的用法.}

Here is an inline note.\footnote{Inlines notes are easier to write,
  since you don't have to pick an identifier and move down to type the
  note.}

學習協同產品設計流程與環境的基本原理與架構.

學習如何在網際專案管理系統的協助下, 進行協同產品設計.

學習如何在協同設計流程中, 進行有效率的工程設計表達與產品資料管理.

延續程式語言與電腦輔助設計實習課程,
學習如何建構協同產品設計環境所需的工具.

\section{Pandoc 手冊}\label{pandoc-ux624bux518a}

http://johnmacfarlane.net/pandoc/README.html

footnote

tables

flexible ordered lists

definition lists

fenced code blocks

superscript

subscript

strikeout

title blocks

automatic tables of contents

embedded LaTeX math

citations

markdown inside HTML block elements

\section{協同產品設計實習專案(2ag1)}\label{ux5354ux540cux7522ux54c1ux8a2dux8a08ux5be6ux7fd2ux5c08ux68482ag1}

\subsection{組員}\label{ux7d44ux54e1}

40123101

40123102

40123132

OpenShift 網站: http://2014cdag1-cadp13ag6.rhcloud.com/

\subsection{w8考試}\label{w8ux8003ux8a66}

\begin{enumerate}
\def\labelenumi{\arabic{enumi}.}
\itemsep1pt\parskip0pt\parsep0pt
\item
  請寫一個執行時可以列出 9×9 乘法表的網際 Python 程式, 然後 Push 到個人
  bitbucket 空間, 而且同步指到 OpenShift 個人帳號上執行.
\end{enumerate}

\begin{itemize}
\item
  程式碼:

\begin{Shaded}
\begin{Highlighting}[]
    \KeywordTok{for} \NormalTok{x in }\DataTypeTok{range}\NormalTok{(}\DecValTok{0}\NormalTok{,}\DecValTok{10}\NormalTok{):}
        \KeywordTok{for} \NormalTok{y in }\DataTypeTok{range}\NormalTok{(}\DecValTok{0}\NormalTok{,}\DecValTok{10}\NormalTok{):}
            \DataTypeTok{print}\NormalTok{(x,}\StringTok{'*'}\NormalTok{,y,}\StringTok{'='}\NormalTok{,x*y)}
\end{Highlighting}
\end{Shaded}
\item
  解題過程:

  打完程式後,之後我把它上傳到bitbucket,git add .→ git commit -m
  ``99''→ git push,之後為了要把99乘法表顯示在openshift
  CMSimply上,進去openshift CMSimply的資料夾→ wsgi資料夾→
  application檔案,在裡面新增一個class 的格式,之後連線FileZilla ,
  把application檔案覆蓋到app-root/runtime/repo/wsgi的application檔案下,之後進去openshift
  CMSimply的網站,就顯示99乘法表了。
\item
  解題心得:

  雖然要打一個99乘法表的程式很快,但要如何顯示在遠端的網頁上,思考了很久,在課堂後請教了TA,最後應用cherrypy的方式,在application檔案,新增一個class
  的格式,就完成了。
\end{itemize}

\begin{enumerate}
\def\labelenumi{\arabic{enumi}.}
\setcounter{enumi}{1}
\itemsep1pt\parskip0pt\parsep0pt
\item
  請將上述執行過程錄為 flv 後, 上傳到個人的 Vemeo 空間中,
  並將網址回報到各組網站 (dokuwiki 與 CMSimply)與報告中,
  並且將相關心得與報告連結登錄到 wiki.mde.tw 第八週的分組頁面中.
\end{enumerate}

\begin{itemize}
\item
  Bitbucket 連結: https://bitbucket.org/40123102/40123102cmsimply/src
\item
  Vemeo 空間: https://vimeo.com/user26935042/videos
\item
  copy 空間: https://copy.com/XohJLdUBRJdk
\item
  dokuwiki網站:
  https://40123102cdg1dokuwiki-cadp13ag6.rhcloud.com/doku.php?id=start
\item
  CMSimply網址: http://40123102cdg1-cadp13ag6.rhcloud.com/w8test/
\end{itemize}

\begin{enumerate}
\def\labelenumi{\arabic{enumi}.}
\setcounter{enumi}{2}
\itemsep1pt\parskip0pt\parsep0pt
\item
  請在各組的雲端 dowiki 中,根據下列 40 個帳號與密碼,
  新增對應的使用者帳號與密碼後, 將雲端網址登錄在 wiki.mde.tw
  各組第八週頁面中, 並說明操作過程與心得後, 將心得整理成 pdf 後繳交到
  course@mde.tw.
\end{enumerate}

\begin{itemize}
\item
  解題過程:

  程式做不出來,只好一個一個建立。
\item
  解題心得:

  想很久,還是無法想出程式,只好用最勤勞的方法,一個一個建立,唉!!!
\end{itemize}

\subsection{w9考試}\label{w9ux8003ux8a66}

\begin{enumerate}
\def\labelenumi{\arabic{enumi}.}
\itemsep1pt\parskip0pt\parsep0pt
\item
  請寫一個執行時可以列出以十為底對數表的網際 Python 程式, 然後 Push
  到個人 bitbucket 空間, 而且同步指到 OpenShift 個人帳號上執行.
\end{enumerate}

\begin{itemize}
\item
  程式碼:

\begin{Shaded}
\begin{Highlighting}[]
\CharTok{import} \NormalTok{math}
\NormalTok{count = }\DecValTok{0}
\NormalTok{text=}\StringTok{'}\OtherTok\DataTypeTok{round}\NormalTok{(math.log(i/}\DecValTok{100}\NormalTok{,}\DecValTok{10}\NormalTok{)*}\DecValTok{10000}\NormalTok{,}\DecValTok{0}\NormalTok{),end=}\StringTok{'-'}\NormalTok{)}
    \NormalTok{count += }\DecValTok{1}
    \KeywordTok{if}\NormalTok{(count == }\DecValTok{10}\NormalTok{):}
        \DataTypeTok{print}\NormalTok{()}
        \NormalTok{count = }\DecValTok{0}
\end{Highlighting}
\end{Shaded}
\item
  解題過程:

  打完程式後,有顯示出來,但無法顯示表單在CMSimply網址上,只能一個一個慢慢打。
\item
  解題心得:

  想了很久,還是無法,只會打程式,只好把顯示出來的數字一個一個加上,慢慢打。
\end{itemize}

\begin{enumerate}
\def\labelenumi{\arabic{enumi}.}
\setcounter{enumi}{1}
\itemsep1pt\parskip0pt\parsep0pt
\item
  請在個人的 OpenShift
  平台上建立一個能夠列印出與九九乘法表結果完全相同的網際程式,
  接著在乘法表上端加上兩個輸入表單, 讓使用者輸入兩個整數, 按下送出鍵後,
  程式會列出以此兩個整數為基底的乘法表, 例如: 若兩個欄位都輸入: 9,
  則列出九九乘法表, 若輸入 9, 20, 則列出 9×20 的乘法表.
\end{enumerate}

\begin{itemize}
\item
  解題過程:

  打完程式後,之後我把它上傳到bitbucket,git add .→ git commit -m
  ``w9\_2''→ git push,之後為了要把99乘法表顯示在openshift
  CMSimply上,進去openshift CMSimply的資料夾→ wsgi資料夾→
  application檔案,在裡面新增一個class 的格式,之後連線FileZilla ,
  把application檔案覆蓋到app-root/runtime/repo/wsgi的application檔案下,之後進去openshift
  CMSimply的網站,輸入 9, 20, 則顯示 9×20 的乘法表.
\item
  解題心得:

  這題跟上週小考差不多,只是還要再新增兩個輸入表單,所以還要在回想一下,但大致上ok,所以完成了。
\end{itemize}

\begin{enumerate}
\def\labelenumi{\arabic{enumi}.}
\setcounter{enumi}{2}
\itemsep1pt\parskip0pt\parsep0pt
\item
  請在各組的雲端 dokuwiki 中, 新增帳號與密碼都是由 abc001
  \textasciitilde{} abc399 字串所組成的 399 名用戶登入對應資料,
  並將製作過程與驗證流程拍成 flv 後上傳到個人的 Vimeo 資料區,
  並將連結放在個人第九週頁面.
\end{enumerate}

\begin{itemize}
\item
  程式碼:

\begin{Shaded}
\begin{Highlighting}[]
\CharTok{import} \NormalTok{hashlib}
\CommentTok{# convert user_password into sha1 encoded string}
\KeywordTok{def} \NormalTok{gen_password(user_password):}
    \KeywordTok{return} \NormalTok{hashlib.sha1(user_password.encode(}\StringTok{"utf-8"}\NormalTok{)).hexdigest()}
\NormalTok{text=}\StringTok{'abc}\OtherTok{%03d}\StringTok{'}
\KeywordTok{for} \NormalTok{i in }\DataTypeTok{range}\NormalTok{(}\DecValTok{1}\NormalTok{,}\DecValTok{399+1}\NormalTok{):}
    \DataTypeTok{print}\NormalTok{(text%(i)+}\StringTok{':'}\NormalTok{+gen_password(text%(i))+}\StringTok{':'}\NormalTok{+text%(i)+}\StringTok{'@gmail.com:'}\NormalTok{+text%(i)+}\StringTok{'@gmail.com:'}\NormalTok{+}\StringTok{'user'}\NormalTok{)}
\end{Highlighting}
\end{Shaded}
\item
  操作過程:

  先寫一個程式使他能產生帳號與密碼都是由 abc001 \textasciitilde{} abc399
  字串所組成的 399 名用戶登入對應資料,之後連線到openshift
  dokuwiki的FileZilla,把產生出來的程式碼貼到app-root/data/conf/users.auth.php下,之後開啟dokuwiki網站即可。
\item
  心得:

  一開始這個程式需要思考一下,如何使帳號與密碼都是由 abc001
  \textasciitilde{} abc399 字串所組成的 399
  名用戶,打出來後還要去讓所產生的程式碼符合一開始的格式,如同users.auth.php,思考很久最後終於成功了。
\end{itemize}

\subsection{w12}\label{w12}

\begin{itemize}
\itemsep1pt\parskip0pt\parsep0pt
\item
  第十二週任務:
\end{itemize}

\begin{enumerate}
\def\labelenumi{\arabic{enumi}.}
\itemsep1pt\parskip0pt\parsep0pt
\item
  請各組將第八週與第九週考試的摘要報告放入 Github
  協同專案中的分組報告區, 並將內容放入各組控管的同步 OpenShift 網站中.
  (佔期末成績 5分)
\item
  請各組設法利用 CherryPy 與 Pro/Web.Link 技術, 在 Github
  協同專案中建立一個能夠透過連結或表單控制 Cube 零件, a, b, 或 c
  零件尺寸的網際協同程式, 讓使用者可以藉以利用近端的 Creo
  嵌入式瀏覽器控制 Cube 的尺寸後列出該零件的體積大小. (佔期末成績 5分)
\end{enumerate}

\begin{itemize}
\itemsep1pt\parskip0pt\parsep0pt
\item
  心得:
\end{itemize}

\begin{enumerate}
\def\labelenumi{\arabic{enumi}.}
\itemsep1pt\parskip0pt\parsep0pt
\item
  在URL直接更改參數:在對應的程式內,把迴圈改成可在URL後面輸入變數,EX:http://127.0.0.1:8080/cdag1/cube1?w=20\&h=20\&l=20
\item
  在更改FOR迴圈時切記\{\}!
  在此DEF最下面還有一個括弧要刪除,否則會跑出CATCH
\end{enumerate}

\begin{itemize}
\item
  組員自評:

  40123101:85分

  40123102:95分

  40123132:95分
\end{itemize}

\subsection{w13}\label{w13}

\begin{itemize}
\itemsep1pt\parskip0pt\parsep0pt
\item
  第十三週任務:
\end{itemize}

\begin{enumerate}
\def\labelenumi{\arabic{enumi}.}
\item
  請以手動方式, 從下方下載 lego\_man.7z 零件, 完成組立後, 以組員為單位,
  在各分組報告區詳細說明組立過程的重點與注意事項. (5 分)
\item
  請利用 Vimeo 及 copy.com 存放任務一中的手動組立流程影片檔案,
  以組員為單位, 將連結放入各分組報告區中. (5 分)
\item
  請仔細觀察協同程式中的 lego\_man 自動組立程式, 並且從下方
  lego\_parts\_edu.7z 中下載相同名稱的 lego\_man 零件,
  請詳細比較兩個壓縮檔案解開的 lego\_man 零件檔案, 有哪些差別?
  請將詳細比對結果以組為單位將資料放入各分組的報告中. (5 分)
\end{enumerate}

\begin{itemize}
\itemsep1pt\parskip0pt\parsep0pt
\item
  40123101心得:
\end{itemize}

任務1:

\begin{enumerate}
\def\labelenumi{\arabic{enumi}.}
\item
  組立時如果使用自動選項,很容易出現組立不成功,譬如你需要重合時他可能會出現相切、距離\ldots{}等,所以如果已知道自己想要甚麼選項時可先自己選擇。
\item
  在組立時建議可將位置拉到你想要的位置附近,才不會有跑掉的狀況。
\end{enumerate}

任務2:

\begin{enumerate}
\def\labelenumi{\arabic{enumi}.}
\item
  Vemeo 空間: https://vimeo.com/95823817
\item
  copy 空間: https://copy.com/z8hxVmer0fvE
\end{enumerate}

\begin{itemize}
\itemsep1pt\parskip0pt\parsep0pt
\item
  40123102心得:
\end{itemize}

任務1:

\begin{enumerate}
\def\labelenumi{\arabic{enumi}.}
\item
  先開啟creo,建立組立檔,依序叫出零件圖的身體、右手、左手、waist、右腳、左腳、頭、帽子,每叫出一個就把它們限制約束,軸和軸或平面和平面,即可完成組裝。
\item
  要注意組立順序以及約束的條件。
\end{enumerate}

任務2:

\begin{enumerate}
\def\labelenumi{\arabic{enumi}.}
\item
  Vemeo 空間: https://vimeo.com/user26935042/videos
\item
  copy 空間: https://copy.com/EOsEOAiVn9Vn
\end{enumerate}

\begin{itemize}
\itemsep1pt\parskip0pt\parsep0pt
\item
  40123132心得:
\end{itemize}

任務1:

\begin{enumerate}
\def\labelenumi{\arabic{enumi}.}
\itemsep1pt\parskip0pt\parsep0pt
\item
  手動組樂高零組件,身體每個部位都要用不同關係組起來,有時候組完他角度還會跑掉,之後研究過程都覺得很有趣,大家都笑的很開心,雖然有點生氣。如果組立過程有錯誤或者是刪除的話數值會跑掉。
\end{enumerate}

任務2:

\begin{enumerate}
\def\labelenumi{\arabic{enumi}.}
\item
  Vemeo 空間: https://vimeo.com/upload
\item
  copy 空間: https://www.copy.com/browse/copy-inbox
\end{enumerate}

\begin{itemize}
\itemsep1pt\parskip0pt\parsep0pt
\item
  第1組心得:
\end{itemize}

任務3:

\begin{enumerate}
\def\labelenumi{\arabic{enumi}.}
\item
  lego\_man 自動組立程式組出來零件的特徵
  ID是按照順序組裝的(40\textasciitilde{}49);但lego\_parts\_edu.7z
  中下載相同名稱的 lego\_man 零件的特徵
  ID卻不是按照順序組裝,而是有跳號的現象,這表示他很有可能不是一次就組立起來,在組立過程中,可能有刪除某些零件之類的,已導致沒有按照順序組立。
\item
  使用自動組立時較迅速也較準確,只需了解程式內容並修改,還有確認零件黨有放對地方即可。
\end{enumerate}

\subsection{w14}\label{w14}

\begin{itemize}
\itemsep1pt\parskip0pt\parsep0pt
\item
  第十四週任務:
\end{itemize}

\begin{enumerate}
\def\labelenumi{\arabic{enumi}.}
\item
  請以 Creo 開啟 remsub6.asm,
  本任務要求各組員以手動組立完成此一零件組立. (5 分)
\item
  請將任務 1 手動組立流程拍成 flv 檔案後, 分別上傳到 copy.com 與 Vimeo
  網站, 並將個人心得與相關連結放入各組協同報告中. (5 分)
\item
  請修改上週小人偶自動組立程式, 以組為單位, 將組立命名為 remsub6 函式,
  當使用者連結至各組雲端網站中的 remsub6, 可以完成自動組立, 操作完成後,
  請將分組心得與相關連結放入各組協同報告中.(5 分)
\end{enumerate}

\section{網際鼓式煞車設計(2ag2)}\label{ux7db2ux969bux9f13ux5f0fux715eux8ecaux8a2dux8a082ag2}

有關鼓式煞車

\section{程式設計架構}\label{ux7a0bux5f0fux8a2dux8a08ux67b6ux69cb}

鼓式煞車

\section{結果與討論}\label{ux7d50ux679cux8207ux8a0eux8ad6}

這裡是結果與討論

\section{W12任務(2ag3)}\label{w12ux4efbux52d92ag3}

請各組將第九週考試的摘要報告放入 Github 協同專案中的分組報告區

\section{摘要報告}\label{ux6458ux8981ux5831ux544a}

\begin{verbatim}
第一題:
\end{verbatim}

請寫一個執行時可以列出以十為底對數表的網際 Python 程式,然後 Push 到個人
bitbucket 空間,而且同步指到 OpenShift 個人帳號上執行。

\begin{Shaded}
\begin{Highlighting}[]
\CharTok{import} \NormalTok{math}
\NormalTok{count = }\DecValTok{0}
\NormalTok{text=}\StringTok{'}\OtherTok\DataTypeTok{round}\NormalTok{(math.log(i/}\DecValTok{100}\NormalTok{,}\DecValTok{10}\NormalTok{)*}\DecValTok{10000}\NormalTok{,}\DecValTok{0}\NormalTok{),end=}\StringTok{'-'}\NormalTok{)}
    \NormalTok{count += }\DecValTok{1}
    \KeywordTok{if}\NormalTok{(count == }\DecValTok{10}\NormalTok{):}
        \DataTypeTok{print}\NormalTok{()}
        \NormalTok{count = }\DecValTok{0}
\end{Highlighting}
\end{Shaded}

\begin{verbatim}
解題心得:
\end{verbatim}

可利用程式算出對數的值,就不需要一個一個按計算機。

\begin{verbatim}
第二題:
\end{verbatim}

請在個人的 OpenShift
平台上建立一個能夠列印出與九九乘法表結果完全相同的網際程式,接著在乘法表上端加上兩個輸入表單,讓使用者輸入兩個整數,按下送出鍵後,程式會列出以此兩個整數為基底的乘法表,例如:若兩個欄位都輸入:
9,則列出九九乘法表,若輸入 9,20,則列出 9×20 的乘法表。

\begin{Shaded}
\begin{Highlighting}[]
\KeywordTok{class} \NormalTok{Example(}\DataTypeTok{object}\NormalTok{):}
\NormalTok{_cp_config = \{}
\CommentTok{# if there is no utf-8 encoding, no Chinese input available}
\StringTok{'tools.encode.encoding'}\NormalTok{: }\StringTok{'utf-8'}\NormalTok{,}
\CommentTok{'tools.sessions.on'} \NormalTok{: }\OtherTok{True}\NormalTok{,}
\CommentTok{'tools.sessions.st\textbackslash{}orage_type'} \NormalTok{: }\StringTok{'file'}\NormalTok{,}
\CommentTok{'tools.sessions.locking'} \NormalTok{: }\StringTok{'explicit'}\NormalTok{,}
\CommentTok{'tools.sessions.storage_path'} \NormalTok{: data_dir+}\StringTok{'/tmp'}\NormalTok{,}
\CommentTok{# session timeout is 60 minutes}
\CommentTok{'tools.sessions.timeout'} \NormalTok{: }\DecValTok{60}
\NormalTok{\}}

\OtherTok{@cherrypy.expose}
\KeywordTok{def} \NormalTok{index(}\OtherTok{self}\NormalTok{):}
    \NormalTok{output = }\StringTok{''}

    \NormalTok{form = }\StringTok{'''}
\StringTok{    <form action='action'>}
\StringTok{    num1:<INPUT type='text' name='num1'>}
\StringTok{    num2:<INPUT type='text'  name='num2'>}
\StringTok{    <input type=submit>}
\StringTok{    <input type=reset>}
\StringTok{    </form>}
\StringTok{    '''}
    \NormalTok{output += form}
    \KeywordTok{return} \NormalTok{output}
\OtherTok{@cherrypy.expose}
\KeywordTok{def} \NormalTok{action(}\OtherTok{self}\NormalTok{, num1=}\DecValTok{9}\NormalTok{, num2=}\DecValTok{9}\NormalTok{):}
    \NormalTok{num1 = }\DataTypeTok{int}\NormalTok{(num1)}
    \NormalTok{num2 = }\DataTypeTok{int}\NormalTok{(num2)}
    \NormalTok{output = }\StringTok{''}
    \KeywordTok{for} \NormalTok{i in }\DataTypeTok{range}\NormalTok{(num1):}
        \KeywordTok{for} \NormalTok{j in }\DataTypeTok{range}\NormalTok{(num2):}
            \NormalTok{output += }\DataTypeTok{str}\NormalTok{(i) + }\StringTok{'*'} \NormalTok{+ }\DataTypeTok{str}\NormalTok{(j) + }\StringTok{'='} \NormalTok{+ }\DataTypeTok{str}\NormalTok{(i*j) + }\StringTok{'<br />'}
    \KeywordTok{return} \NormalTok{output}
\end{Highlighting}
\end{Shaded}

解題心得:

利用網站輸入值並算出,並需要有兩個輸入表單。

\begin{verbatim}
第三題:
\end{verbatim}

請在各組的雲端 dokuwiki 中,新增帳號與密碼都是由 abc001
\textasciitilde{} abc399 字串所組成的 399
名用戶登入對應資料,並將製作過程與驗證流程拍成 flv 後上傳到個人的 Vimeo
資料區,並將連結放在個人第九週頁面。

\begin{Shaded}
\begin{Highlighting}[]
\CharTok{import} \NormalTok{hashlib}
\CommentTok{#convert user_password into sha1 encoded string}
\KeywordTok{def} \NormalTok{gen_password(user_password):}
    \KeywordTok{return} \NormalTok{hashlib.sha1(user_password.encode(}\StringTok{"utf-8"}\NormalTok{)).hexdigest()}
\NormalTok{text=}\StringTok{'abc}\OtherTok{%03d}\StringTok{'}
\KeywordTok{for} \NormalTok{i in }\DataTypeTok{range}\NormalTok{(}\DecValTok{1}\NormalTok{,}\DecValTok{399+1}\NormalTok{):}
    \DataTypeTok{print}\NormalTok{(text%(i)+}\StringTok{':'}\NormalTok{+gen_password(text%(i))+}\StringTok{':'}\NormalTok{+text%(i)+}\StringTok{'@gmail.com:'}\NormalTok{+text%(i)+}\StringTok{'@gmail.com:'}\NormalTok{+}\StringTok{'user'}\NormalTok{)}
\end{Highlighting}
\end{Shaded}

解題心得:

可以利用程式創造多人帳密,一起管理網站,減少一個一個創建帳密,又會有被盜用的風險。

import math count = 0 text=`\%04d' for i in range(100,200+10):
print(text\%round(math.log(i/100,10)*10000,0),end=`-') count += 1
if(count == 10): print() count = 0

class Example(object): \emph{cp}config = \{ \# if there is no utf-8
encoding, no Chinese input available `tools.encode.encoding': `utf-8',
`tools.sessions.on' : True, `tools.sessions.storage\_type' : `file',
`tools.sessions.locking' : `explicit', `tools.sessions.storage\_path' :
data\_dir+`/tmp', \# session timeout is 60 minutes
`tools.sessions.timeout' : 60 \}

@cherrypy.expose def index(self): output = ''

\begin{verbatim}
form = '''
<form action='action'>
num1:<INPUT type='text' name='num1'>
num2:<INPUT type='text'  name='num2'>
<input type=submit>
<input type=reset>
</form>
'''
output += form
return output
\end{verbatim}

@cherrypy.expose def action(self, num1=9, num2=9): num1 = int(num1) num2
= int(num2) output = `' for i in range(num1): for j in range(num2):
output += str(i) +'\emph{`+ str(j) +'=' + str(i}j) + `' return output

import hashlib \#convert user\_password into sha1 encoded string def
gen\_password(user\_password): return
hashlib.sha1(user\_password.encode(``utf-8'')).hexdigest()
text=`abc\%03d' for i in range(1,399+1):
print(text\%(i)+`:'+gen\_password(text\%(i))+`:'+text\%(i)+`@gmail.com:'+text\%(i)+`@gmail.com:'+`user')

\section{網際鼓式煞車設計(2ag4)}\label{ux7db2ux969bux9f13ux5f0fux715eux8ecaux8a2dux8a082ag4}

github連結 40123107: https://github.com/40123107 40123120:
https://github.com/40123120 cdag4:
https://github.com/2014cdag4/2014cdag4

cmsimply連結 40123107: https://github.com/40123107 40123120:
https://github.com/40123120 40123150: http://cdg4-40123150.rhcloud.com/
===

第九週 1.(第一題 30\%) 請寫一個執行時可以列出以十為底對數表的網際 Python
程式, 然後 Push 到個人 bitbucket 空間, 而且同步指到 OpenShift
個人帳號上執行.
http://cdg4-40123150.rhcloud.com/get\_page?heading=\%E7\%AC\%AC\%E4\%B8\%80\%E9\%A1\%8C
2.(第二題 40\%) 請在個人的 OpenShift
平台上建立一個能夠列印出與九九乘法表結果完全相同的網際程式,
接著在乘法表上端加上兩個輸入表單, 讓使用者輸入兩個整數, 按下送出鍵後,
程式會列出以此兩個整數為基底的乘法表, 例如: 若兩個欄位都輸入: 9,
則列出九九乘法表, 若輸入 9, 20, 則列出 9×20 的乘法表.
http://cdg4-40123150.rhcloud.com/example/ http://vimeo.com/92577008
3.(第三題 30\%) 請在各組的雲端 dokuwiki 中, 新增帳號與密碼都是由 abc001
\textasciitilde{} abc399 字串所組成的 399 名用戶登入對應資料,
並將製作過程與驗證流程拍成 flv 後上傳到個人的 Vimeo 資料區,
並將連結放在個人第九週頁面.
https://php-40123150.rhcloud.com/doku.php?id=start
http://vimeo.com/92577072 === w12 請連結
https://github.com/2014cdag4/2014cdag4 ===

這裡是結果與討論

網際協同產品設計(2ag5)
期末報告網站:https://2014cdag5-yimin40123157.rhcloud.com ===

◎協同流程架構: 1. 說明:善用各網站所提供的資源,
進行期末專案全班協同管理 2. 近端 USB/Portable hard
drive:儲存文件檔案資料 \# Leo 程式管理, Pandoc 自動產生文件 3. 遠端
Openshift 網站:免費最多能建立 3 個網頁 4. 遠端 Github
倉儲:專案無限制人數協同, 免費倉儲只能 Public 5. 遠端 Bitbucket
倉儲:免費專案最多 5 人協同, 倉儲能變更 Public/Privacy ===

◎Pro/​Web.Link: 1. 確認近端協同環境版次 2. Leo 4.11, Creo win7 64位元
3. Creo Pro/​Web.Link 透過 javascript 執行 \#Creo only IE 4. 設定 Creo
``File → Options → Configuration Editor'' 底下 5.
web\_enable\_javascript 設為 ``on'' 6. regen\_failure\_handling 設為
``resolve\_mode'' 7. 設定 IE 瀏覽器 ``網際網路選項'' 底下 8.``安全性 →
✔信任的網站 → 網站 → 新增Pro/​Web.Link網站'' 9.``自訂等級 →
ActiveX控制項與外掛程式'' 所有選項設為 ``●啟用'' ===

◎git 衝突解決: 1. git pull 2. git reset 3. git add -A 4. git status 5.
git commit -m ``fix problems'' 6. git push

by 2014cdag5

\section{協同產品設計實習(cdag6)}\label{ux5354ux540cux7522ux54c1ux8a2dux8a08ux5be6ux7fd2cdag6}

組員:40123109黃芯庭、40123119呂冠緯、40123142楊佳致
小組openshift網站:http://2014cdag6-40123109cd2014.rhcloud.com
小組github網站:https://github.com/2014cdag9/2014cdag6

=== *第八週考試內容

(第一題) 請寫一個執行時可以列出 9×9 乘法表的網際 Python 程式, 然後 Push
到個人 bitbucket 空間, 而且同步指到 OpenShift 個人帳號上執行.

(解題過程) \#程式碼 for x in range(0,10): for y in range(0,10):
print(x,'\emph{`,y,'=',x}y)

(第二題) 請將上述執行過程錄為 flv 後, 上傳到個人的 Vemeo 空間中,
並將網址回報到各組網站 (dokuwiki 與 CMSimply)與報告中,
並且將相關心得與報告連結登錄到 wiki.mde.tw 第八週的分組頁面中.

(第三題) (協同計分, 分組進行) 請在各組的雲端 dowiki 中,根據下列 40
個帳號與密碼, 新增對應的使用者帳號與密碼後, 將雲端網址登錄在 wiki.mde.tw
各組第八週頁面中, 並說明操作過程與心得後, 將心得整理成 pdf 後繳交到
course@mde.tw.

*第九週考試內容

(第一題 30\%) 請寫一個執行時可以列出以十為底對數表的網際 Python 程式,
然後 Push 到個人 bitbucket 空間, 而且同步指到 OpenShift 個人帳號上執行.
(解題過程) \#程式碼 python import math count = 0 text=`\%04d' for i in
range(100,200+10):
print(text\%round(math.log(i/100,10)*10000,0),end=`-') count += 1
if(count == 10): print() count = 0 \#openshift連結
https://2014s-40123109.rhcloud.com/get\_page?heading=\%C2\%A0\%E7\%AC\%AC\%E4\%B9\%9D\%E9\%80\%B1

(第二題 40\%) 請在個人的 OpenShift
平台上建立一個能夠列印出與九九乘法表結果完全相同的網際程式,
接著在乘法表上端加上兩個輸入表單, 讓使用者輸入兩個整數, 按下送出鍵後,
程式會列出以此兩個整數為基底的乘法表, 例如: 若兩個欄位都輸入: 9,
則列出九九乘法表, 若輸入 9, 20, 則列出 9×20 的乘法表. (解題過程)
\#openshift連結 http://2014s-40123109.rhcloud.com/example/
\#bitbucket連結 https://bitbucket.org/40123109/40123109cdg6

(第三題 30\%) 請在各組的雲端 dokuwiki 中, 新增帳號與密碼都是由 abc001
\textasciitilde{} abc399 字串所組成的 399 名用戶登入對應資料,
並將製作過程與驗證流程拍成 flv 後上傳到個人的 Vimeo 資料區,
並將連結放在個人第九週頁面. (解題過程) \#雲端dokuwiki頁面
http://wiki-40123109.rhcloud.com/doku.php \#vimeo連結
https://vimeo.com/92575709

*第十二週任務 將第八週與第九週考試的摘要報告放入 Github
協同專案中的分組報告區, 並將內容放入各組控管的同步 OpenShift 網站中

*第十三週任務 任務一: 請以手動方式, 從下方下載 lego\_man.7z 零件,
完成組立後, 以組員為單位,
在各分組報告區詳細說明組立過程的重點與注意事項. (5 分)

任務二: 請利用 Vimeo 及 copy.com 存放任務一中的手動組立流程影片檔案,
以組員為單位, 將連結放入各分組報告區中. (5 分)

任務三: 請仔細觀察協同程式中的 lego\_man 自動組立程式, 並且從下方
lego\_parts\_edu.7z 中下載相同名稱的 lego\_man 零件,
請詳細比較兩個壓縮檔案解開的 lego\_man 零件檔案, 有哪些差別?
請將詳細比對結果以組為單位將資料放入各分組的報告中. (5 分)

\section{網際鼓式煞車設計(2ag7)}\label{ux7db2ux969bux9f13ux5f0fux715eux8ecaux8a2dux8a082ag7}

有關鼓式煞車

\section{程式設計架構}\label{ux7a0bux5f0fux8a2dux8a08ux67b6ux69cb-1}

鼓式煞車

\section{結果與討論}\label{ux7d50ux679cux8207ux8a0eux8ad6-1}

這裡是結果與討論

\section{網際鼓式煞車設計(2ag8)}\label{ux7db2ux969bux9f13ux5f0fux715eux8ecaux8a2dux8a082ag8}

第八組

組員:黃柏良 王若嘉

第八週

(第二題) 請將上述執行過程錄為 flv 後, 上傳到個人的 Vemeo 空間中,
並將網址回報到各組網站 (dokuwiki 與 CMSimply)與報告中,
並且將相關心得與報告連結登錄到 wiki.mde.tw 第八週的分組頁面

(第三題) (協同計分, 分組進行) 請在各組的雲端 dowiki 中,根據下列 40
個帳號與密碼, 新增對應的使用者帳號與密碼後, 將雲端網址登錄在 wiki.mde.tw
各組第八週頁面中, 並說明操作過程與心得後, 將心得整理成 pdf 後繳交到
course@mde.tw. 第九週

(第一題 30\%) 請寫一個執行時可以列出以十為底對數表的網際 Python 程式,
然後 Push 到個人 bitbucket 空間, 而且同步指到 OpenShift 個人帳號上執行.
(解題過程)

第一題

\section{程式碼}\label{ux7a0bux5f0fux78bc}

python import math count = 0 text=`\%04d' for i in range(100,200+10):
print(text\%round(math.log(i/100,10)*10000,0),end=`-') count += 1
if(count == 10): print() count = 0 \#openshift連結
https://2014s-40123109.rhcloud.com/get\_page?heading=\%C2\%A0\%E7\%AC\%AC\%E4\%B9\%9D\%E9\%80\%B1

第二題

請在個人的 OpenShift
平台上建立一個能夠列印出與九九乘法表結果完全相同的網際程式,
接著在乘法表上端加上兩個輸入表單, 讓使用者輸入兩個整數, 按下送出鍵後,
程式會列出以此兩個整數為基底的乘法表, 例如: 若兩個欄位都輸入: 9,
則列出九九乘法表, 若輸入 9, 20, 則列出 9×20 的乘法表.

http://cdg8-40123141.rhcloud.com/example/

https://bitbucket.org/40123141/---

第三題

請在各組的雲端 dokuwiki 中, 新增帳號與密碼都是由 abc001
\textasciitilde{} abc399 字串所組成的 399 名用戶登入對應資料,
並將製作過程與驗證流程拍成 flv 後上傳到個人的 Vimeo 資料區,
並將連結放在個人第九週頁面.

http://wiki-40123141.rhcloud.com/doku.php

http://vimeo.com/92589824

第十二週

將第八週與第九週考試的摘要報告放入 Github 協同專案中的分組報告區,
並將內容放入各組控管的同步 OpenShift 網站中

第十三週

任務一: 請以手動方式, 從下方下載 lego\_man.7z 零件, 完成組立後,
以組員為單位, 在各分組報告區詳細說明組立過程的重點與注意事項. (5 分)

任務二: 請利用 Vimeo 及 copy.com 存放任務一中的手動組立流程影片檔案,
以組員為單位, 將連結放入各分組報告區中. (5 分)

任務三: 請仔細觀察協同程式中的 lego\_man 自動組立程式, 並且從下方
lego\_parts\_edu.7z 中下載相同名稱的 lego\_man 零件,
請詳細比較兩個壓縮檔案解開的 lego\_man 零件檔案, 有哪些差別?
請將詳細比對結果以組為單位將資料放入各分組的報告中. (5 分

\section{協同產品設計實習(cdag9)}\label{ux5354ux540cux7522ux54c1ux8a2dux8a08ux5be6ux7fd2cdag9}

組員:40123114王瑞豪 組員:40123117吳昱緯 組員:40123127洪宗佑

小組openshift網站:https://2014cadg9-40123114.rhcloud.com/
小組github網站:https://github.com/coursemdetw/2014cda

=== W8考試

1.請寫一個執行時可以列出 9×9 乘法表的網際 Python 程式, 然後 Push 到個人
bitbucket 空間, 而且同步指到 OpenShift 個人帳號上執行.

解題

// 設定一個0到10範圍,y也設定0到10,接著印出x,'\emph{`,y,'=',x}y // 程式碼

\begin{verbatim}
for x in range(0,10):
    for y in range(0,10):
        print(x,'*',y,'=',x*y)
\end{verbatim}

\begin{enumerate}
\def\labelenumi{\arabic{enumi}.}
\setcounter{enumi}{1}
\item
  請將上述執行過程錄為 flv 後, 上傳到個人的 Vemeo 空間中,
  並將網址回報到各組網站 (dokuwiki 與 CMSimply)與報告中,
  並且將相關心得與報告連結登錄到 wiki.mde.tw 第八週的分組頁面中.
\item
  (協同計分, 分組進行) 請在各組的雲端 dowiki 中,根據下列 40
  個帳號與密碼, 新增對應的使用者帳號與密碼後, 將雲端網址登錄在
  wiki.mde.tw 各組第八週頁面中, 並說明操作過程與心得後, 將心得整理成 pdf
  後繳交到 course@mde.tw.
\end{enumerate}

這題不太會做 只能夠一個一個慢慢輸入

W9考試

1.請寫一個執行時可以列出以十為底對數表的網際 Python 程式, 然後 Push
到個人 bitbucket 空間, 而且同步指到 OpenShift 個人帳號上執行. 解題

程式碼 python import math count = 0 text=`\%04d' for i in
range(100,200+10):
print(text\%round(math.log(i/100,10)*10000,0),end=`-') count += 1
if(count == 10): print() count = 0 openshift連結
https://2014s-40123114.rhcloud.com/

\begin{enumerate}
\def\labelenumi{\arabic{enumi}.}
\setcounter{enumi}{1}
\itemsep1pt\parskip0pt\parsep0pt
\item
  請在個人的 OpenShift
  平台上建立一個能夠列印出與九九乘法表結果完全相同的網際程式,
  接著在乘法表上端加上兩個輸入表單, 讓使用者輸入兩個整數, 按下送出鍵後,
  程式會列出以此兩個整數為基底的乘法表, 例如: 若兩個欄位都輸入: 9,
  則列出九九乘法表, 若輸入 9, 20, 則列出 9×20 的乘法表. 解題
  \#openshift連結 http://2014s-40123114.rhcloud.com/example/
  \#bitbucket連結 https://bitbucket.org/40123114/40123114cdg9
\end{enumerate}

心得: 這題跟上一週的很相似,只是變更了一下範圍,又學習到了新的東西

(第三題 30\%)3.abc399 字串所組成的 399 名用戶登入對應資料,
並將製作過程與驗證流程拍成 flv 後上傳到個人的 Vimeo 資料區,
並將連結放在個人第九週頁面. 解題

http://wiki-40123114.rhcloud.com/doku.php \#vimeo連結
https://vimeo.com/92575092

心得:
第三題學習到如何用程式去輸入對應資料,不必在一個一個利用手動方式再輸入了

W12 將第八週與第九週考試的摘要報告放入 Github 協同專案中的分組報告區,
並將內容放入各組控管的同步 OpenShift 網站中

\section{cd2ag10報告(2ag10)}\label{cd2ag10ux5831ux544a2ag10}

以下為各週報告

\section{第十組組員(2ag10)}\label{ux7b2cux5341ux7d44ux7d44ux54e12ag10}

小組網站:http://goo.gl/aztMHi

40123116-吳羽閔

40123118-吳謦麟

40123153-戴志軒

\section{第八週考試摘要(2ag10)}\label{ux7b2cux516bux9031ux8003ux8a66ux6458ux89812ag10}

小組網站:http://goo.gl/yi3qJ5

2014S CD Week8

課程教材:

A collaborative writing approach to wikis

Collabrative engine for distributed mechanical design

Web-based collaborative engineering support system

第八週考試題目

下列題目完成後, 必須(1)將程式碼送到個人的 Bitbucket repository
下(2)程式可在近端與雲端部署執行(3)在 wiki.mde.tw 個人第八週心得中留下
Bitbucket 與雲端執行連結(4)整理出一份小考第一(二, 或三)題的 PDF 檔案,
寄到 course@mde.tw, 標題為: 學號-小考第一(二, 或三)題,
內容必須包含程式碼, 解題過程, 解題心得, Bitbucket 連結, 雲端網址等資料.

(第一題) 請寫一個執行時可以列出 9×9 乘法表的網際 Python 程式, 然後 Push
到個人 bitbucket 空間, 而且同步指到 OpenShift 個人帳號上執行.

(第二題) 請將上述執行過程錄為 flv 後, 上傳到個人的 Vemeo 空間中,
並將網址回報到各組網站 (dokuwiki 與 CMSimply)與報告中,
並且將相關心得與報告連結登錄到 wiki.mde.tw 第八週的分組頁面中.

(第三題) (協同計分, 分組進行) 請在各組的雲端 dowiki 中,根據下列 40
個帳號與密碼, 新增對應的使用者帳號與密碼後, 將雲端網址登錄在 wiki.mde.tw
各組第八週頁面中, 並說明操作過程與心得後, 將心得整理成 pdf 後繳交到
course@mde.tw.

40 個帳號與密碼

58B39 , 9J35UAVM

3624D , QANF34CW

7345B , 3PAFXKWZ

358DA , E6RJFKW4

3BC7B , HG2ASNYH

CA55C , XUZTHWQK

D2756 , UHK2W3D2

368B8 , A5QGYA6W

7948A , PF278WDQ

C65C4 , GT4KBCXU

A8964 , 3RPQSW2U

22422 , 7E57K7F3

9A5B4 , 4MVKRE5Z

B5A76 , 3DTAAHUF

57388 , NP39FGXR

8A833 , C7DNBHCQ

29AB8 , 6KMGK73Z

5ABD2 , PV5FH722

86293 , PJ69FBMS

9DCBC , U5HR6QR8

276DD , URE9FNWD

599AA , X2P6CTXF

9C449 , DKRN3V59

64236 , 86UWN3E9

43AAC , MNCJZCAX

73B93 , QX945VJJ

36283 , M3MQGUXD

7447C , TQZVDKPT

C73AB , MSP4GPPX

8284B , XMT8W9RD

62454 , SD4C7V89

44B3C , 636DBRJC

C75CB , M66RMMQ2

4A3CB , FF485EQ4

7D248 , FGJHQDAS

A7CC4 , R47AHA4Y

99BCC , RNQYZGQZ

6DA77 , HRCDP9D8

357AB , CKB4Q2EC

A9525 , JDVX75ST

第八週協同設計練習題目:

建立各組 OpenShift 上的 dokuwiki

http://ethercalc.tw/ Openshift 上的 ethercalc

近端的 ethercalc

用 markdown 編輯電子書: https://github.com/progit/progit ,

http://johnmacfarlane.net/pandoc/epub.html

期中考試與分組報告必選題

請以三組共 9 個人的情況下(座位為 3×3 配置),
分析2014s\_week3的協同設計題目, 如何配置座位可以得到團隊的最大配分.
假如將組別擴大為五組共 15 人的情況下(座位為 4×4 配置,
則團隊座位安排後的最大配合又是多少?

上課影片 http://vimeo.com/user24079973/videos

\section{第八週報告(2ag10)}\label{ux7b2cux516bux9031ux5831ux544a2ag10}

小組網站:http://goo.gl/YCNd7p

\begin{Shaded}
\begin{Highlighting}[]
\CharTok{import} \NormalTok{cherrypy}
\KeywordTok{class} \NormalTok{HelloWorld(}\DataTypeTok{object}\NormalTok{):}
    \OtherTok{@cherrypy.expose}
    \KeywordTok{def} \NormalTok{index(}\OtherTok{self}\NormalTok{, var1=}\DecValTok{9}\NormalTok{, var2=}\DecValTok{9}\NormalTok{):}
        \CommentTok{# initialize outstring}
        \NormalTok{outstring = }\StringTok{""}
        \CommentTok{# initialize count}
        \NormalTok{count = }\DecValTok{0}
        \NormalTok{d = }\DataTypeTok{int}\NormalTok{(var1)}
        \NormalTok{e = }\DataTypeTok{int}\NormalTok{(var2)+}\DecValTok{1}
        \KeywordTok{for} \NormalTok{i in }\DataTypeTok{range}\NormalTok{(}\DecValTok{1}\NormalTok{, d):}
            \KeywordTok{for} \NormalTok{j in }\DataTypeTok{range}\NormalTok{(}\DecValTok{1}\NormalTok{, e):}
                \NormalTok{count += }\DecValTok{1}
                \CommentTok{#print(count)}
                \KeywordTok{if} \NormalTok{count%(}\DataTypeTok{int}\NormalTok{(var2)) == }\DecValTok{0}\NormalTok{:}
                    \NormalTok{outstring += }\StringTok{"<td>"}\NormalTok{+}\DataTypeTok{str}\NormalTok{(i) + }\StringTok{"*"} \NormalTok{+ }\DataTypeTok{str}\NormalTok{(j) + }\StringTok{"="} \NormalTok{+ }\DataTypeTok{str}\NormalTok{(i*j) +}\StringTok{"</td>"}\NormalTok{+}\StringTok{"</tr>"} \NormalTok{+ }\StringTok{"<br />"}
                \KeywordTok{else}\NormalTok{:}
                    \NormalTok{outstring += }\StringTok{"<td>"}\NormalTok{+}\DataTypeTok{str}\NormalTok{(i) + }\StringTok{"x"} \NormalTok{+ }\DataTypeTok{str}\NormalTok{(j) + }\StringTok{"="} \NormalTok{+ }\DataTypeTok{str}\NormalTok{(i*j) + }\StringTok{"</td>"}
        \KeywordTok{return} \StringTok{"<table border=2><tr><td>99乘法表</td></tr><tr>"}\NormalTok{+outstring}
\CommentTok{#http://127.0.0.1:8080/index?var1=10&var2=20}
\CommentTok{#"&nbsp;"*4}
 
 
\NormalTok{cherrypy.quickstart(HelloWorld())}
\CommentTok{#application=cherrypy.Application(HelloWorld()) #將符號打開上傳openshift即可使用,並刪掉上列}
\end{Highlighting}
\end{Shaded}

以上程式碼即可跑出99乘法表.

9x9openshift: http://9x9-cadp13ag8.rhcloud.com/
(因openshift空間不足,故利用第九週程式,程式碼略有差別,主要為多新增兩個輸入表單,其餘相同)

同步bitbucket: https://bitbucket.org/40123153/input9x9 (同上)

------------我是分隔線------------

影片(由於網路超慢,怕錄製影片過大所以只錄製最後結果)

https://vimeo.com/92003966

P.S 由於題目理解錯誤,所以上述影片為失效。

\section{第九週考試摘要(2ag10)}\label{ux7b2cux4e5dux9031ux8003ux8a66ux6458ux89812ag10}

小組網站:http://goo.gl/8nPx2Y

2014S CD Week9

課程教材:

A collaborative writing approach to wikis

Collabrative engine for distributed mechanical design

Web-based collaborative engineering support system

期中考試題目

下列題目完成後, 必須(1)將程式碼送到個人的 Bitbucket repository
下(2)程式可在近端與雲端部署執行(3)在 wiki.mde.tw 個人第九週心得中留下
Bitbucket 與雲端執行連結以及其它參考連結(4)整理出一份期中考第一(二,
或三)題的 PDF 檔案, 寄到 course@mde.tw, 標題為:
cda\_學號\_姓名\_期中考第一(二, 或三)題(乙班將 cda 改為 cdb),
內容必須包含程式碼, 解題過程, 解題心得, Bitbucket 連結, 雲端網址等資料.

(第一題 30\%) 請寫一個執行時可以列出以十為底對數表的網際 Python 程式,
然後 Push 到個人 bitbucket 空間, 而且同步指到 OpenShift 個人帳號上執行.

(第二題 40\%) 請在個人的 OpenShift
平台上建立一個能夠列印出與九九乘法表結果完全相同的網際程式,
接著在乘法表上端加上兩個輸入表單, 讓使用者輸入兩個整數, 按下送出鍵後,
程式會列出以此兩個整數為基底的乘法表, 例如: 若兩個欄位都輸入: 9,
則列出九九乘法表, 若輸入 9, 20, 則列出 9×20 的乘法表.

(第三題 30\%) 請在各組的雲端 dokuwiki 中, 新增帳號與密碼都是由 abc001
\textasciitilde{} abc399 字串所組成的 399 名用戶登入對應資料,
並將製作過程與驗證流程拍成 flv 後上傳到個人的 Vimeo 資料區,
並將連結放在個人第九週頁面.

第一題參考資料: 對數表與應用, 當 x=10, 對應到 0 行的值, 表示要對 1.00
取以十為底的對數, 所得到的值為 math.log(1.00, 10)=0, 而 x=10 對應到 1
行的值, 表示要對 1.01 取以十為底的對數, 所得到的值為 math.log(1.01,
10)=0.004321373782642578 然後再乘上 10000, 並且只取整數,
所以對應表的值為 43, 也就是表中的 0043, 當 x=20, 而且對應到第 9 行的值,
則為 math.log(2.09, 10)=0.32014628611105395, 然後再乘上 10000, 只取整數,
所以對應表上的值為 3201.

第一題參考: Python 中 str() 可以將整數或浮點數轉為字串, int()
則可以將字串轉為整數, round(1.0123, 2) 表示只取小數點後兩位, math.log(x,
10) 表示對 x 取以 10 為底的 log 值, html 表格請參考.

期中成績評量

請各組依據2014s\_week7中的說明完成期中報告.(最後繳交期限為 2014.4.26
晚上 12:00)

成績評量時, 將依據各組在 wiki.mde.tw 中的頁面進行評量,
期中成績包括平時成績(參考個人自評成績)、第八週考試與期中考試成績、期中報告成績等.

上課影片 http://vimeo.com/user24079973/videos

\section{第九週報告(2ag10)}\label{ux7b2cux4e5dux9031ux5831ux544a2ag10}

小組網站:http://goo.gl/Hy7Ktn

第一題:

Log以10為底之表單openshift: http://log-cadp13ag8.rhcloud.com/

bitbucket: https://bitbucket.org/40123153/log10-1.00-1.99

因為迴圈關係,多了一個21\ldots{}還在更新中。(解決)

P.S已利用elif判斷式將最後的21給剃除!

當在最後20的地方下elif將

給剃除,故不在往下做表格及列印出21數字。

\begin{Shaded}
\begin{Highlighting}[]
\CharTok{import} \NormalTok{cherrypy}
\CharTok{import} \NormalTok{os}
\CharTok{import} \NormalTok{math}
\CommentTok{# 1. 導入所需模組}

\CommentTok{# 2. 設定近端與遠端目錄}
\CommentTok{# 確定程式檔案所在目錄, 在 Windows 有最後的反斜線}
\NormalTok{_curdir = os.path.join(os.getcwd(), os.path.dirname(}\OtherTok{__file__}\NormalTok{))}
\CommentTok{# 設定在雲端與近端的資料儲存目錄}
\KeywordTok{if} \StringTok{'OPENSHIFT_REPO_DIR'} \NormalTok{in os.environ.keys():}
    \CommentTok{# 表示程式在雲端執行}
    \NormalTok{download_root_dir = os.environ[}\StringTok{'OPENSHIFT_DATA_DIR'}\NormalTok{]}
    \NormalTok{data_dir = os.environ[}\StringTok{'OPENSHIFT_DATA_DIR'}\NormalTok{]}
\KeywordTok{else}\NormalTok{:}
    \CommentTok{# 表示程式在近端執行}
    \NormalTok{download_root_dir = _curdir + }\StringTok{"/local_data/"}
    \NormalTok{data_dir = _curdir + }\StringTok{"/local_data/"}
 
\CommentTok{# 3. 建立主物件}
\KeywordTok{class} \NormalTok{HelloWorld(}\DataTypeTok{object}\NormalTok{):}
    \OtherTok{@cherrypy.expose}
    \KeywordTok{def} \NormalTok{index2(}\OtherTok{self}\NormalTok{, input1=}\OtherTok{None}\NormalTok{, input2=}\OtherTok{None}\NormalTok{):}
        \KeywordTok{return} \StringTok{"Hello world!"}\NormalTok{+}\DataTypeTok{str}\NormalTok{(input1)}
    \OtherTok{@cherrypy.expose}
    \KeywordTok{def} \NormalTok{inputform(}\OtherTok{self}\NormalTok{, input1=}\OtherTok{None}\NormalTok{, input2=}\OtherTok{None}\NormalTok{):}
        \KeywordTok{return} \StringTok{"input form"}\NormalTok{+}\DataTypeTok{str}\NormalTok{(input1)}
    \CommentTok{#index.exposed = True}
    \OtherTok{@cherrypy.expose}
    \KeywordTok{def} \NormalTok{index(}\OtherTok{self}\NormalTok{):}
        \NormalTok{out=}\StringTok{""}
        \NormalTok{c = }\DecValTok{0}
        \NormalTok{k = }\DecValTok{0}
        \KeywordTok{for} \NormalTok{i in }\DataTypeTok{range}\NormalTok{(}\DecValTok{0}\NormalTok{,}\DecValTok{11}\NormalTok{):}
            \KeywordTok{for} \NormalTok{j in }\DataTypeTok{range} \NormalTok{(}\DecValTok{0}\NormalTok{,}\DecValTok{10}\NormalTok{):}
                \NormalTok{k += }\DecValTok{1}
                \NormalTok{c = }\DecValTok{10} \NormalTok{+ i}
                \NormalTok{d = c/}\DecValTok{10}\NormalTok{+j/}\DecValTok{100}
                \NormalTok{e = (}\DataTypeTok{int}\NormalTok{(math.log(d,}\DecValTok{10}\NormalTok{)*}\DecValTok{10000}\NormalTok{))}
                \KeywordTok{if} \NormalTok{(k%}\DecValTok{10} \NormalTok{== }\DecValTok{0} \NormalTok{and k<=}\DecValTok{100}\NormalTok{):}
                    \NormalTok{out += }\StringTok{"<td>"}\NormalTok{+}\DataTypeTok{str}\NormalTok{(e)+}\StringTok{"</td></tr><td>"}\NormalTok{+}\DataTypeTok{str}\NormalTok{(c}\DecValTok{+1}\NormalTok{)+}\StringTok{"</td><br />"}
                \KeywordTok{elif} \NormalTok{k%}\DecValTok{100} \NormalTok{== }\DecValTok{0}\NormalTok{:}
                    \NormalTok{out += }\StringTok{"<td>"}\NormalTok{+}\DataTypeTok{str}\NormalTok{(e)+}\StringTok{"</td><br />"}
                \KeywordTok{else}\NormalTok{:}
                    \NormalTok{out +=}\StringTok{"<td>"}\NormalTok{+}\DataTypeTok{str}\NormalTok{(e) + }\StringTok{"</td>"}
                \CommentTok{#print("log=",int(math.log(d,10)*10000))}
            \CommentTok{#print("\textbackslash{}n")}
        \KeywordTok{return} \StringTok{"<table border=1><tr><td>x</td><td>0</td><td>1</td><td>2</td><td>3</td><td>4</td><td>5</td><td>6</td><td>7</td><td>8</td><td>9</td></tr><td>"}\NormalTok{+}\StringTok{"10"}\NormalTok{+}\StringTok{"</td>"}\NormalTok{+out}
 
\CommentTok{# 4. 安排啟動設定}
\CommentTok{# 配合程式檔案所在目錄設定靜態目錄或靜態檔案}
\NormalTok{application_conf = \{}\StringTok{'/static'}\NormalTok{:\{}
        \StringTok{'tools.staticdir.on'}\NormalTok{: }\OtherTok{True}\NormalTok{,}
        \CommentTok{'tools.staticdir.dir'}\NormalTok{: _curdir+}\StringTok{"/static"}\NormalTok{\},}
        \CommentTok{'/downloads'}\NormalTok{:\{}
        \StringTok{'tools.staticdir.on'}\NormalTok{: }\OtherTok{True}\NormalTok{,}
        \CommentTok{'tools.staticdir.dir'}\NormalTok{: data_dir+}\StringTok{"/downloads"}\NormalTok{\}}
    \NormalTok{\}}
 
\CommentTok{# 5. 在近端或遠端啟動程式}
\CommentTok{# 利用 HelloWorld() class 產生案例物件}
\NormalTok{root = HelloWorld()}
\CommentTok{# 假如在 os 環境變數中存在 'OPENSHIFT_REPO_DIR', 表示程式在 OpenShift 環境中執行}
\KeywordTok{if} \StringTok{'OPENSHIFT_REPO_DIR'} \NormalTok{in os.environ.keys():}
    \CommentTok{# 雲端執行啟動}
    \NormalTok{application = cherrypy.Application(root, config = application_conf)}
\KeywordTok{else}\NormalTok{:}
    \CommentTok{# 近端執行啟動}
    \CommentTok{'''}
\CommentTok{    cherrypy.server.socket_port = 8083}
\CommentTok{    cherrypy.server.socket_host = '127.0.0.1'}
\CommentTok{    '''}
    \NormalTok{cherrypy.quickstart(root, config = application_conf)}
\end{Highlighting}
\end{Shaded}

------------我是分隔線------------

第二題:

9×9乘法表,並且有兩input可以改變9×9乘法表的範圍。例:input:9 and 20
則為9×20乘法表。

openshift: http://9x9-cadp13ag8.rhcloud.com/ (備註:i為前值,j為後值)

bitbucket: https://bitbucket.org/40123153/input9x9

\begin{Shaded}
\begin{Highlighting}[]
\CharTok{import} \NormalTok{cherrypy}
\KeywordTok{class} \NormalTok{HelloWorld(}\DataTypeTok{object}\NormalTok{):}
    \OtherTok{@cherrypy.expose}
    \KeywordTok{def} \NormalTok{index(}\OtherTok{self}\NormalTok{, var1=}\DecValTok{9}\NormalTok{, var2=}\DecValTok{9}\NormalTok{):}
        \CommentTok{# initialize outstring}
        \NormalTok{outstring = }\StringTok{""}
        \CommentTok{# initialize count}
        \NormalTok{count = }\DecValTok{0}
        \NormalTok{d = }\DataTypeTok{int}\NormalTok{(var1)+}\DecValTok{1}
        \NormalTok{e = }\DataTypeTok{int}\NormalTok{(var2)+}\DecValTok{1}
        \KeywordTok{for} \NormalTok{i in }\DataTypeTok{range}\NormalTok{(}\DecValTok{1}\NormalTok{, d):}
            \KeywordTok{for} \NormalTok{j in }\DataTypeTok{range}\NormalTok{(}\DecValTok{1}\NormalTok{, e):}
                \NormalTok{count += }\DecValTok{1}
                \KeywordTok{if} \NormalTok{count%}\DataTypeTok{int}\NormalTok{(var2) == }\DecValTok{0}\NormalTok{:}
                    \NormalTok{outstring += }\StringTok{"<td>"}\NormalTok{+}\DataTypeTok{str}\NormalTok{(i) + }\StringTok{"*"} \NormalTok{+ }\DataTypeTok{str}\NormalTok{(j) + }\StringTok{"="} \NormalTok{+ }\DataTypeTok{str}\NormalTok{(i*j) +}\StringTok{"</td>"}\NormalTok{+}\StringTok{"</tr>"} \NormalTok{+ }\StringTok{"<br />"}
                \KeywordTok{else}\NormalTok{:}
                    \NormalTok{outstring += }\StringTok{"<td>"}\NormalTok{+}\DataTypeTok{str}\NormalTok{(i) + }\StringTok{"x"} \NormalTok{+ }\DataTypeTok{str}\NormalTok{(j) + }\StringTok{"="} \NormalTok{+ }\DataTypeTok{str}\NormalTok{(i*j) + }\StringTok{"</td>"}
        \KeywordTok{return} \StringTok{"<table border=3><tr><td>99乘法表</td></tr><tr>"}\NormalTok{+outstring+}\StringTok{'''<br/><form method="POST" action="index">}
\StringTok{                i 我是前值:<input type="text" name="var1"><br />}
\StringTok{                j 我是後值:<input type="text" name="var2"><br />}
\StringTok{                <input type="submit" value="send">}
\StringTok{                </form>}
\StringTok{                '''}
\CommentTok{#http://127.0.0.1:8080/index?var1=10&var2=20}
 
 
\CommentTok{#cherrypy.quickstart(HelloWorld())}
\NormalTok{application=cherrypy.Application(HelloWorld())}
\end{Highlighting}
\end{Shaded}

------------我是分隔線------------

第三題:

新增帳號與密碼都是由 abc001 \textasciitilde{} abc399 字串所組成的 399。

影片解說: https://vimeo.com/92574166

首先,我們必須要先有abc001\textasciitilde{}abc399的帳號密碼值,故先用程式迴圈跑出。
(因為懶得寫輸出檔,或者是不熟悉所以不冒險去寫。)

將值複製至文件內將空白修掉,利用EXCEL把密碼在複製至B欄位。存成CSV檔

同樣利用文件將逗點修掉(如果沒記錯空白處為一個TAB,為了安全還是複製之前的。)

再利用程式,修改讀取的文件,將密碼處改為數列\href{數列第一資料行位為{[}0{]},第二資料故為{[}1{]}}{1}

跑出user之後利用FZ將檔案覆蓋,則可以在雲端上登錄。(為了確認完全傳入,將abc001改為管理權限,確認帳號輸入)

dokuwiki\_openshift: http://dokuwiki-cadp13ag8.rhcloud.com/doku.php

bitbucket: https://bitbucket.org/40123153/dokuwiki\_abc

第三題的程式碼實在不知道要打什麼\ldots{}.

\section{第十二週報告(2ag10)}\label{ux7b2cux5341ux4e8cux9031ux5831ux544a2ag10}

小組網站:http://goo.gl/28PW52

第十二週任務

請各組將第八週與第九週考試的摘要報告放入 Github 協同專案中的分組報告區,
並將內容放入各組控管的同步 OpenShift 網站中. (佔期末成績 5分)

github小組倉儲:https://github.com/2014cdag10

github大分組倉儲:https://github.com/coursemdetw/2014cda

請各組設法利用 CherryPy 與 Pro/Web.Link 技術, 在 Github
協同專案中建立一個能夠透過連結或表單控制 Cube 零件, a, b, 或 c
零件尺寸的網際協同程式, 讓使用者可以藉以利用近端的 Creo 嵌入式瀏覽器控制
Cube 的尺寸後列出該零件的體積大小. (佔期末成績 5分)

\begin{enumerate}
\def\labelenumi{\arabic{enumi}.}
\itemsep1pt\parskip0pt\parsep0pt
\item
  Creo 必須使用 64 位元版本 (配合 Windows 7 操作系統)
\item
  Creo web\_enable\_javascript 設為 on, regen\_failure\_handling 設為
  resolve\_mode
\item
  IE→工具→網際網路選項→安全性→信任的網站→自訂等級 允許信任網站執行
  ActiveX
\item
  cda 專案必須將http://cdag10-40123153.rhcloud.com/設為信任網站
\item
  開啟 Creo 2.0, 建立一個 cube, relations: d0 = a, d1 = b, d2 = c (a, b,
  c 為 local parameters)
\item
  然後在嵌入式 IE, 連接到
  http://cdag10-40123153.rhcloud.com/cdag10/cube10
  (\#在parameter中將函式名稱改為cube10)
\item
  執行 http://cdag10-40123153.rhcloud.com/cdag10/fourbar10
  (\#在parameter中將函式名稱改為fourbar10)
  之前則需要先下載四連桿零件放在 V:\home\fourbar 目錄中,
  並開啟一個空的組立檔案, 執行時 Pro/Web.Link 程式會自動進行連桿組立.
\item
  Pro/Web.Link 應用:
  http://www.ptc.com/company/news/inprint/taiwan/proe5.htm
\item
  參考資料: http://inversionconsulting.blogspot.tw/search/label/WebLink
\end{enumerate}

首先必須將creo 2.0 的
config.pro檔做些設定,必須要將regen\_failure\_handing設為resolv\_mode,使他能重繪。

再來必須將web\_enable\_javascript設為on,這樣才能讓javascript可以動作。

#很重要!!這將會成為成功與失敗的原因\ldots{}.

並且將IE關於ActiveX都通通啟用,將http://cdag10-40123153.rhcloud.com/設為例外網頁。

再來就是繪圖,將零件的三個尺寸設關係式,d0=a , d1=b , d2=c。

所有前置動作都完成了,再來就是利用creo裡面內建的IE跑囉!

將http://cdag10-40123153.rhcloud.com/cdag10/cube10貼至IE的url中,就會開始跑囉!

(原始尺寸為150x150x150的方塊)

第一次:3375000

第二次:4096000

第三次:4913000

第四次:5832000

第五次:6859000

大致上就這樣吧\ldots{}不過 c變數
忘記動到\ldots{}不過只是照a與b同樣的型式\ldots{}將a或b複製下來改成c就可以三軸同動了。

\section{第十二週評分(2ag10)}\label{ux7b2cux5341ux4e8cux9031ux8a55ux52062ag10}

40123116 : 10分

40123118 : 10分

40123153 : 10分

\section{第十三週考試(2ag10)}\label{ux7b2cux5341ux4e09ux9031ux8003ux8a662ag10}

小組網站:http://goo.gl/til1Nq

https://vimeo.com/95823422 copy.com: https://copy.com/P9fGFaLSlQqx

https://vimeo.com/95823554 copy.com: https://copy.com/1pnPvVduOJ9B

任務一: 請以手動方式, 從下方下載 lego\_man.7z 零件, 完成組立後,
以組員為單位, 在各分組報告區詳細說明組立過程的重點與注意事項. (5 分)

任務二: 請利用 Vimeo 及 copy.com 存放任務一中的手動組立流程影片檔案,
以組員為單位, 將連結放入各分組報告區中. (5 分)

任務三: 請仔細觀察協同程式中的 lego\_man 自動組立程式, 並且從下方
lego\_parts\_edu.7z 中下載相同名稱的 lego\_man 零件,
請詳細比較兩個壓縮檔案解開的 lego\_man 零件檔案, 有哪些差別?
請將詳細比對結果以組為單位將資料放入各分組的報告中. (5 分)

主要任務: 了解各組員對於課程內容的了解 (採每一組員分別完成任務進行測試)

協同環境與架構

實際操作

Creo 零件手動組立

Creo 零件參數手動設定

Creo 零件參數網際變更

Creo 零件自動組立

lego\_man.7z

lego\_parts\_edu.7z

weblink user guide.pdf

\section{第十三週報告(2ag10)}\label{ux7b2cux5341ux4e09ux9031ux5831ux544a2ag10}

小組網站:http://goo.gl/QFJp5F

因為只有說利用creo,但並沒限定使用Parametric或Direct.
所以利用Direct必較人性化\ldots{}所以用Direct組立(主要是沒有順序問題!)

首先載入身體,放置在任意座標.

在載入左、右手臂,利用同心圓及重合將其組上.(注意面的位子,將要組的面先轉至要組上的面的方向)

之後將手載入兩根,利用同心圓及重合組在手臂前端.

將下盤組在身體上,利用三個平面的重合即可.

完成後將左、右腳利用同心圓及重合組在下盤兩端.(注意面的位子,將要組的面先轉至要組上的面的方向)

載入頭部,同樣利用同心圓及重合組在身體上(注意孔大小)

載入帽子,利用三個重合即可(第一個是曲面的重合,中間平面的重合、下平面的重合)

小組vedio: https://vimeo.com/95822338

小組copy.com: http://goo.gl/K1nh1G

 lego\_man 自動組立程式組出來的Feat ID是順序的40 41 42
\ldots{}一個一個零件載入裝配.  而lego\_parts\_edu.7z 中下載相同名稱的
lego\_man 零件卻不是這樣順序的\ldots{}
 這代表他並不是一次就組起來,可能途中有改變、刪除什麼的.

自動組立的某些限制條件或許沒訂得很好,動作有點好笑!

P.S 自動組立不支援creo 2.0
M70(x64)版本,會跑出錯誤!只可以用於M60(x64)版本

P.S2 無法將man.py程式移植到我們(cdag10)小組網頁中,也就是說無法利用
https://cdag10-40123153.rhcloud.com/cdag10/man/開啟\ldots{}

更無法利用https://cdag10-40123153.rhcloud.com/cdag10/man/assembly來跑自動組立.

(可以利用https://cdag10-40123153.rhcloud.com/cdag30/man/assembly
來跑自動組立。)

\section{第十三週評分(2ag10)}\label{ux7b2cux5341ux4e09ux9031ux8a55ux52062ag10}

40123116 : 15分

40123118 : 15分

40123153 : 15分

\section{網際鼓式煞車設計(2ag11)}\label{ux7db2ux969bux9f13ux5f0fux715eux8ecaux8a2dux8a082ag11}

有關鼓式煞車

\section{程式設計架構}\label{ux7a0bux5f0fux8a2dux8a08ux67b6ux69cb-2}

鼓式煞車

\section{結果與討論}\label{ux7d50ux679cux8207ux8a0eux8ad6-2}

這裡是結果與討論

\section{網際鼓式煞車設計(2ag12)}\label{ux7db2ux969bux9f13ux5f0fux715eux8ecaux8a2dux8a082ag12}

有關鼓式煞車

\section{程式設計架構}\label{ux7a0bux5f0fux8a2dux8a08ux67b6ux69cb-3}

鼓式煞車

\section{結果與討論}\label{ux7d50ux679cux8207ux8a0eux8ad6-3}

這裡是結果與討論

\section{協同產品設計實習專案(2ag13)}\label{ux5354ux540cux7522ux54c1ux8a2dux8a08ux5be6ux7fd2ux5c08ux68482ag13}

\subsection{組員:}\label{ux7d44ux54e1-1}

40123155

40123124

40123138

OpenShift 網站:
https://python-40123124.rhcloud.com/get\_page?heading=\%3Cspan\%3E\%E6\%9C\%9F\%E4\%B8\%AD\%E8\%80\%83\%3C/span\%3E

第十二週

請各組將第八週與第九週考試的摘要報告放入 Github 協同專案中的分組報告區,
並將內容放入各組控管的同步 OpenShift 網站中. (佔期末成績 5分)

考試摘要 http://dokuwiki-40123155.rhcloud.com/doku.php

第十三週

40123124
一開始組合的時候不太熟悉,因為不常使用creo這個繪圖軟體,可是慢慢的熟能生巧之後,
一下子就完成了,只是過程有許多小細節要注意,
像是面和面、軸和軸的對應要正確,如此一來,接合就會非常順利。

40123138 第一次接觸到零件組立,相當不熟悉,
但是摸索了一下子就會將LogoMan已條件限制的方式接合起來,過程中覺得還滿有趣的。

40123155 CREO組件時要把相關的接起 要非常注意設定定位
因為找錯會無法連接正確 組起樂高機器人後 對於CREO的使用 如何應用組合
有點了解 重點就是不要找錯條線找錯面 就成功一半 剩下的就是自己判斷正確
不要失誤 就可組合成功

\begin{itemize}
\item
  bitbucket網站: https://40123155@bitbucket.org/40123155/---.git
\item
  dokuwiki網站: http://dokuwiki-40123155.rhcloud.com
\item
  vimeo網站: https://vimeo.com/user26959367
\item
  CMSimply 網站: https://python-40123124.rhcloud.com
\item
  下載網站: https://copy.com/SjKQ8LHZpVFq
\item
  HTML網站:
  http://dokuwiki-40123155.rhcloud.com/lib/exe/fetch.php?media=\%E6\%9C\%9F\%E4\%B8\%AD\%E5\%A0\%B1\%E5\%91\%8A.pdf
\end{itemize}

http://dokuwiki-40123155.rhcloud.com/lib/exe/fetch.php?media=\%E8\%AA\%B2\%E7\%A8\%8B\%E6\%95\%99\%E6\%9D\%90\%E4\%B8\%80\%E7\%BF\%BB\%E8\%AD\%AF\_.pdf

http://dokuwiki-40123155.rhcloud.com/lib/exe/fetch.php?media=\%E8\%AA\%B2\%E7\%A8\%8B\%E6\%95\%99\%E6\%9D\%90\%E4\%BA\%8C\%E7\%BF\%BB\%E8\%AD\%AF.pdf

組員自評:

\begin{verbatim}
40123155:80分

40123124:70分

40123138:75分
\end{verbatim}

\section{協同產品設計實習專案(2ag14)}\label{ux5354ux540cux7522ux54c1ux8a2dux8a08ux5be6ux7fd2ux5c08ux68482ag14}

w8

第八週考試題目 下列題目完成後, 必須(1)將程式碼送到個人的 Bitbucket
repository 下(2)程式可在近端與雲端部署執行(3)在 wiki.mde.tw
個人第八週心得中留下 Bitbucket 與雲端執行連結(4)整理出一份小考第一(二,
或三)題的 PDF 檔案, 寄到 course@mde.tw, 標題為: 學號-小考第一(二,
或三)題, 內容必須包含程式碼, 解題過程, 解題心得, Bitbucket 連結,
雲端網址等資料.

(第一題) 請寫一個執行時可以列出 9×9 乘法表的網際 Python 程式, 然後 Push
到個人 bitbucket 空間, 而且同步指到 OpenShift 個人帳號上執行

`''

1*1=1

1*2=2

1*3=3

1*4=4

1*5=5

1*6=6

1*7=7

1*8=8

1*9=9

2*1=2

2*2=4

2*3=6

2*4=8

2*5=10

2*6=12

2*7=14

2*8=16

2*9=18

3*1=3

3*2=6

3*3=9

3*4=12

3*5=15

3*6=18

3*7=21

3*8=24

3*9=27

4*1=4

4*2=8

4*3=12

4*4=16

4*5=20

4*6=24

4*7=28

4*8=32

4*9=36

5*1=5

5*2=10

5*3=15

5*4=20

5*5=25

5*6=30

5*7=35

5*8=40

5*9=45

6*1=6

6*2=12

6*3=18

6*4=24

6*5=30

6*6=36

6*7=42

6*8=48

6*9=54

7*1=7

7*2=14

7*3=21

7*4=28

7*5=35

7*6=42

7*7=49

7*8=56

7*9=63

8*1=8

8*2=16

8*3=24

8*4=32

8*5=40

8*6=48

8*7=56

8*8=64

8*9=72

9*1=9

9*2=18

9*3=27

9*4=36

9*5=45

9*6=54

9*7=63

9*8=72

9*9=81

`''

(第二題) 請將上述執行過程錄為 flv 後, 上傳到個人的 Vemeo 空間中,
並將網址回報到各組網站 (dokuwiki 與 CMSimply)與報告中,
並且將相關心得與報告連結登錄到 wiki.mde.tw 第八週的分組頁面中.

(第三題) (協同計分, 分組進行) 請在各組的雲端 dowiki 中,根據下列 40
個帳號與密碼, 新增對應的使用者帳號與密碼後, 將雲端網址登錄在 wiki.mde.tw
各組第八週頁面中, 並說明操作過程與心得後, 將心得整理成 pdf 後繳交到
course@mde.tw.

`'' import hashlib

`'' sha1 secure hashes `'' st\_date ={[}{]}

def gen\_password(user\_password): return
hashlib.sha1(user\_password.encode(``utf-8'')).hexdigest()

with open(``test.txt'', ``wt'') as out\_file: with open(``2014.txt'',
``r'', encoding=``utf-8'')as in\_file: for index in in\_file:

\begin{verbatim}
        number, name =index.strip().split(' ')
        mail=number+"@gm.nfu.edu.tw"
        print(number,name)
        st_date.append([number, name, mail])
        out_file.write(str(number)+":"+gen_password(str(number))+":"+str(mail)+":user"+"\n")
\end{verbatim}

`''

=== w9

期中考試題目 下列題目完成後, 必須(1)將程式碼送到個人的 Bitbucket
repository 下(2)程式可在近端與雲端部署執行(3)在 wiki.mde.tw
個人第九週心得中留下 Bitbucket
與雲端執行連結以及其它參考連結(4)整理出一份期中考第一(二, 或三)題的 PDF
檔案, 寄到 course@mde.tw, 標題為: cda\_學號\_姓名\_期中考第一(二,
或三)題(乙班將 cda 改為 cdb), 內容必須包含程式碼, 解題過程, 解題心得,
Bitbucket 連結, 雲端網址等資料.

(第一題 30\%) 請寫一個執行時可以列出以十為底對數表的網際 Python 程式,
然後 Push 到個人 bitbucket 空間, 而且同步指到 OpenShift個人帳號上執行.

程式 : https://bitbucket.org/40123136/444

(第二題 40\%) 請在個人的 OpenShift
平台上建立一個能夠列印出與九九乘法表結果完全相同的網際程式,
接著在乘法表上端加上兩個輸入表單, 讓使用者輸入兩個整數, 按下送出鍵後,
程式會列出以此兩個整數為基底的乘法表, 例如: 若兩個欄位都輸入: 9,
則列出九九乘法表, 若輸入 9, 20, 則列出 9×20 的乘法表. (第三題 30\%)
請在各組的雲端 dokuwiki 中, 新增帳號與密碼都是由 abc001
\textasciitilde{} abc399 字串所組成的 399 名用戶登入對應資料,
並將製作過程與驗證流程拍成 flv 後上傳到個人的 Vimeo 資料區,
並將連結放在個人第九週頁面.

程式 : https://bitbucket.org/40123136/99

(第三題 30\%) 請在各組的雲端 dokuwiki 中, 新增帳號與密碼都是由 abc001
\textasciitilde{} abc399 字串所組成的 399 名用戶登入對應資料,
並將製作過程與驗證流程拍成 flv 後上傳到個人的Vimeo 資料區,
並將連結放在個人第九週頁面.

Vimeo : https://vimeo.com/92577999

Bitbucket

https://bitbucket.org/40123136/444 https://bitbucket.org/40123136/99

OpenShift

http://cda14-40123136.rhcloud.com/ http://python-cadpag5.rhcloud.com/
https://2014a14-40123136.rhcloud.com/

Vimeo

https://vimeo.com/92577999

=== w13

任務一: 請以手動方式, 從下方下載 lego\_man.7z 零件, 完成組立後,
以組員為單位, 在各分組報告區詳細說明組立過程的重點與注意事項. (5 分)

1.開啟lego\_body.prt(主體)之後接上lego\_arm\_rt.prt、lego\_arm\_lt.prt
2.加上lego\_hand\_lt.prt再來就將機器人的頭加上lego\_head.prt
3.在接上lego\_waist.prt再來就將雙腿接上
4.最後將lego\_head.prt及lego\_hat.prt 機器人組合就完成了,
之後利用爆炸圖可以得到爆炸後的各零件但需要再調整一下位置,就能完成良好的爆炸圖即可完成作業
這是一個組合圖教學珍貴的經驗,一個了解自動組合程式的機會。

任務二: 請利用 Vimeo 及 copy.com 存放任務一中的手動組立流程影片檔案,
以組員為單位, 將連結放入各分組報告區中. (5 分)

 小組vedio: https://vimeo.com/95872800

 小組copy.com: https://copy.com/kJl91O34CfvA

任務三: 請仔細觀察協同程式中的 lego\_man 自動組立程式, 並且從下方
lego\_parts\_edu.7z 中下載相同名稱的 lego\_man 零件,
請詳細比較兩個壓縮檔案解開的 lego\_man 零件檔案, 有哪些差別?
請將詳細比對結果以組為單位將資料放入各分組的報告中. (5 分)

 lego\_man 自動組立程式組出來的Feat ID是40 41 42 的零件載入裝配.
 lego\_parts\_edu.7z 中下載的 lego\_man 零件卻不是這樣順序的\ldots{}
 這代表他不是一次就組起來的

===

這裡是結果與討論

\section{網際鼓式煞車設計(2ag15)}\label{ux7db2ux969bux9f13ux5f0fux715eux8ecaux8a2dux8a082ag15}

有關鼓式煞車

\section{程式設計架構}\label{ux7a0bux5f0fux8a2dux8a08ux67b6ux69cb-4}

鼓式煞車

\section{結果與討論}\label{ux7d50ux679cux8207ux8a0eux8ad6-4}

這裡是結果與討論

\section{網際鼓式煞車設計(2ag16)}\label{ux7db2ux969bux9f13ux5f0fux715eux8ecaux8a2dux8a082ag16}

有關鼓式煞車

\section{程式設計架構}\label{ux7a0bux5f0fux8a2dux8a08ux67b6ux69cb-5}

鼓式煞車

\section{結果與討論}\label{ux7d50ux679cux8207ux8a0eux8ad6-5}

這裡是結果與討論

\section{第八週摘要報告}\label{ux7b2cux516bux9031ux6458ux8981ux5831ux544a}

以協同方式寫Wiki

本文目的在於讓學生利用開發的wiki與應用快速原型的協作式寫作的開發方法。而本文也由定性數據收集和分析方法的來評價。
最後,對於協作方式來寫作的方式,其影響極其討論協議到組別與軟件開發方面的考慮和教學相關的要求問題
研究方法 wiki的協作式寫作,基於設計的研究包括以下四個步驟: 1.
研究的現狀與認識目的並審查與wiki的協作式寫作相關的問題。 2.
wiki將用於促進以協同方式寫作來設計,參與和群體互動。 3.
使用多種方法收集其經驗數據。 4.
通過系統的評價分析,並通過各種方法收集資料。分析,設計,實施和評價是相互共存的。而缺點提出於每個週期提出,重新設計,重新實現,並重新評估。
維基應用域教育在協同方式寫作方法而成的Wiki是適用於多種情況,期望得到的wiki應用在學習投資效益與協同方式來寫作。它十分靈活的,足以適應的專業條件。更具體地,它的應用領域包括不同級別的更高教育,從研究生到社會教育。除了協同寫作與學科相關,維基可以為一些應用程序的開發系統來使用,如產生教材,網路上互相評論,並收集數據於一種項目。

結論和未來研究方向 1.
這項結果不能限制於狹小的研究的範圍,即使調查結果反映了那些報導中的一致性研究文獻。
2.
用來判斷以不同方式來呈現協同,尤其是同組評議,這擁有極高的教育價值與分析性思維,並參與交流。可以促進團體間的合作和時間,wiki打開了協同寫作上新的視野,而小組互動絕不是容易的事情,解決了技術,教學和文化各種問題。
未來的工作將更顯的協同的重要性,wiki以此設計為基礎下更加精進。重要的是,它更進一步影響高等教育對於wiki的使用。最後,它也進一步實踐於系統評論的基礎

\section{第九週摘要報告}\label{ux7b2cux4e5dux9031ux6458ux8981ux5831ux544a}

題目一 :請寫一個執行時可以列出以十為底對數表的網際 Python 程式, 然後
Push 到個人 bitbucket 空間, 而且同步指到 OpenShift 個人帳號上執行.

資料 : https://bitbucket.org/40123158/test

題目三 : 請在個人的 OpenShift
平台上建立一個能夠列印出與九九乘法表結果完全相同的網際程式,
接著在乘法表上端加上兩個輸入表單, 讓使用者輸入兩個整數, 按下送出鍵後,
程式會列出以此兩個整數為基底的乘法表, 例如: 若兩個欄位都輸入: 9,
則列出九九乘法表, 若輸入 9, 20, 則列出 9×20 的乘法表.

資料 : 第17組dokuwiki , 新增 abc001 \textasciitilde{} abc399 用戶資料
http://wikig17-weis.rhcloud.com/doku.php?id=start

上傳影片- 個人Vimeo https://vimeo.com/92577964

Bitbucket 連結 https://bitbucket.org/40123137/week9 Bitbucket 連結
https://bitbucket.org/40123137/week9

協同產品設計實習專案(2ag18)

===============

組員:

40123126 柯嘉洋

40123128 胡瀚中

40123149 蔡東維

===============

OpenShift:

===============

第十三週任務:

任務一:

40123149:

\begin{verbatim}
步驟:建立新的組立檔,後依序步驟逐漸組裝,而插入Lego零件的順序有影響,
    如果胡亂插入會有無法組裝的問題,所以要特別注意,而每次插入零件檔
    時要先調整好位置並藉由基準面與基準面對齊,便可完成。

    解題步驟圖片: https://copy.com/gHzVnqHZnpZt
    
心得:經過這次的試題後,組裝零件要考慮的事情也是很多的。
\end{verbatim}

2ag21

組員: 40023143、40123143、40123145、40123156

本組CMSimply網站: http://cmsimply-40123156.rhcloud.com/

=== 第十三週

任務一: 請以手動方式, 從下方下載 lego\_man.7z 零件, 完成組立後,
以組員為單位, 在各分組報告區詳細說明組立過程的重點與注意事項. (5 分)

任務二: 請利用 Vimeo 及 copy.com 存放任務一中的手動組立流程影片檔案,
以組員為單位, 將連結放入各分組報告區中. (5 分)

任務三: 請仔細觀察協同程式中的 lego\_man 自動組立程式, 並且從下方
lego\_parts\_edu.7z 中下載相同名稱的 lego\_man 零件,
請詳細比較兩個壓縮檔案解開的 lego\_man 零件檔案, 有哪些差別?
請將詳細比對結果以組為單位將資料放入各分組的報告中. (5 分)

=== 第十三週 (40023143)

任務一:
1.開啟lego\_body.prt利用重合將開啟後的lego\_arm\_rt.prt和lego\_arm\_lt.prt接上,再改變角度。
2.開啟lego\_hand\_lt.prt接上lego\_arm\_rt.prt和lego\_arm\_lt.prt完成身體與手
3.開啟lego\_head.prt接在身體上在開啟lego\_hat.prt完成頭的部分
4.開啟lego\_waist.prt接在身體下。
5.最後開啟lego\_leg.prt和lego\_waist.prt接合,完成lego\_men
6.利用爆炸圖的功能將lego\_man爆炸。
心得:利用creo將各個零件組合起來,真的很有趣。
但因為操作不順和錄製程式的畫質調太低,影片不太成功,
但是確實體會了creo的方便,利用雲端將畫好的檔案下載,再利用creo將複雜的零件組合,真的可以大大提升工作效率。

任務二: Vimeo: https://vimeo.com/95867694 copy.com:
https://www.copy.com/s/DEGh0OTCbkDD/3330.flv

任務三: lego\_man 利用自動組立程式組出來的Feat ID的排序是40、41、42
、43\ldots{}\ldots{}以此類推, 一個一個零件檔載入裝配;
而lego\_parts\_edu.7z 中下載相同名稱的 lego\_man 零件檔卻
不是這樣排序的,
表是lego\_man不是一次就組起來,途中應該有改變、刪除\ldots{}\ldots{}等程序。

=== 第十三週 (40123143)

任務一:
首先,先開啟lego\_body.prt(主體)之後接上lego\_arm\_rt.prt、lego\_arm\_lt.prt之後再加上
lego\_hand\_lt.prt再來就將機器人的頭加上lego\_head.prt,在接上lego\_waist.prt再來就將
雙腿接上最後將lego\_head.prt及lego\_hat.prt,機器人組合就完成了,最後利用爆炸圖可以
得到爆炸後的各零件但需要在調整一下位置,就能完成良好爆炸圖即可完成作業,謝謝老師
的教導,這是一個很特別的組合圖教學,希望能快點了解自動組合程式。

=== 第十三週 (40123156)

任務一: 組立過程為 1.開啟creo 2.新建``組立'' 3.用``插入''功能加入零件
4.設定零件間的限制條件 注意事項為 設定零件間的限制條件
要盡量設定到零件完全不能移動

任務二: Vimeo: https://vimeo.com/95857245 copy.com:
https://copy.com/zJSJ6ZVGdtsy

任務三: lego\_parts\_edu.7z裡的樂高零件比較多
可是組合起來卻沒有太大差異,第二個壓
縮檔裡的零件可能是分成更細小的零件。

\section{網際 OpenJSCAD
程式設計(coursemdetw)}\label{ux7db2ux969b-openjscad-ux7a0bux5f0fux8a2dux8a08coursemdetw}

將 Spur 改為凸輪零件成型

\section{設計程式架構}\label{ux8a2dux8a08ux7a0bux5f0fux67b6ux69cb}

定義凸輪設計公式

\section{結果與討論}\label{ux7d50ux679cux8207ux8a0eux8ad6-6}

有關凸倫程式設計的結果與討論

\end{document}
